%# -*- coding:utf-8 -*-
\documentclass[12pt]{article}
\usepackage{graphicx}
\usepackage{amsmath}
\usepackage{geometry}
\usepackage{hyperref}
\usepackage{indentfirst}
\usepackage{amsfonts}

\usepackage[BoldFont,SlantFont,CJKsetspaces,CJKchecksingle]{xeCJK}
\geometry{margin=1in}
\linespread{1.5}

\setCJKmainfont[BoldFont=SimHei]{SimSun}
\setCJKmonofont{SimSun}% 设置缺省中文字体



\begin{document}

\section{牛顿迭代法}
\subsection{方程求根}
首先,选择一个接近函数$f(x)$零点的$x_0$,计算相应的$f(x_0)$和切线斜率$f'(x_0)$(这里$f'$表示函数f的导数)。然后我们计算穿过点$(x_0, f(x_0))$ 并且斜率为$f'(x_0)$的直线和x轴的交点的x坐标,也就是求如下方程的解:

$f(x_0)= (x_0-x)\cdot f'(x_0)$
我们将新求得的点的$x$坐标命名为$x_1$,通常$x_1$会比$x_0$更接近方程$f(x)=0$的解。因此我们现在可以利用$x_1$开始下一轮迭代。迭代公式可化简为如下所示:

$x_{n+1} = x_n - \frac{f(x_n)}{f'(x_n)}$
已经证明,如果$f'$是连续的,并且待求的零点$x$是孤立的,那么在零点x周围存在一个区域,只要初始值$x_0$位于这个邻近区域内,那么牛顿法必定收敛。 并且,如果$f'(x)$不为0, 那么牛顿法将具有平方收敛的性能. 粗略的说,这意味着每迭代一次,牛顿法结果的有效数字将增加一倍。
\subsection{海森矩阵}
在数学中,海森矩阵(Hessian matrix或Hessian)是一个自变量为向量的实值函数的二阶偏导数组成的方块矩阵,此函数如下:

$f(x_1, x_2, \dots, x_n),$
如果$f$所有的二阶导数都存在,那么$f$的海森矩阵即:

$H(f)_{ij}(x) = D_i D_j f(x)$
其中$x = (x_1, x_2, \dots, x_n)$,即
$$
H(f) = \begin{bmatrix}
\frac{\partial^2 f}{\partial x_1^2} & \frac{\partial^2 f}{\partial x_1\,\partial x_2} & \cdots & \frac{\partial^2 f}{\partial x_1\,\partial x_n} \\  \\
\frac{\partial^2 f}{\partial x_2\,\partial x_1} & \frac{\partial^2 f}{\partial x_2^2} & \cdots & \frac{\partial^2 f}{\partial x_2\,\partial x_n} \\  \\
\vdots & \vdots & \ddots & \vdots \\  \\
\frac{\partial^2 f}{\partial x_n\,\partial x_1} & \frac{\partial^2 f}{\partial x_n\,\partial x_2} & \cdots & \frac{\partial^2 f}{\partial x_n^2}
\end{bmatrix}
$$

当函数$f: \mathbb{R}^n \to \mathbb{R}$二阶连续可导时,Hessian矩阵$H$在临界点$x_0$上是一个$n\times n$阶的对称矩阵。

当$H$是正定矩阵时,临界点$x_0$是一个局部的极小值。

当$H$是负定矩阵时,临界点$x_0$是一个局部的极大值。

$H=0$,需要更高阶的导数来帮助判断。

在其余情况下,临界点$x_0$不是局部极值。

\subsection{函数最优化}
给定二阶导数连续的函数$f: \mathbb{R}^2 \to \mathbb{R}$,在($x+\Delta x$)点按Taylar展开,得:

$f(x+\Delta x) \approx Q(x) = f(x) + \nabla f(x)^T \Delta x + \frac{1}{2} \Delta x^T H(x) \Delta x  $  \\
其中$\Delta x = x_{n+1} - x_{n}$,$\nabla f(x)$是$f(x)$的梯度, $H(x)$是Hessian矩阵。

令$\nabla Q(x) = \nabla f(x) + H(x) \Delta x = 0$

得到$H(x) ( x_{n+1} - x_{n} ) = -\nabla f(x)$

所以,n元函数最优化的牛顿迭代公式为:$x_{n+1} = x_{n} - H(x)^{-1} \nabla f(x)$



\end{document}

